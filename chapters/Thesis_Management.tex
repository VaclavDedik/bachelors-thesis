\chapter{Thesis Management}

\section{Introduction and Terminology}

This chapter aims to give a brief description of the thesis management and a coverage of the terminology that will be used in this thesis.

First of all, we should define the term \emph{Thesis Management}. There are two most imported terms that are associated with the thesis management, \emph{thesis topic} and \emph{thesis}. It is important to differentiate between these two terms as the author will often use them throughout the thesis. Thesis topic (or just topic) is, as the term itself suggests, the topic of a thesis. Thesis, on the other hand, is a project of the student that is associated with the thesis topic and is the final part of an advanced university degree. It is clear now that to manage theses, we also need to manage thesis topics. Thesis management is therefore any work that pertains to either the management of theses or the management of thesis topics.

\subsection{Thesis Topic}

When a student wants to graduate from a university, they need to write a long paper on a topic of their choice. To choose a topic, they usually look for it in a list of topics created by either their university or an external party, e.g. a company. If the topic is created by the university, there is a party called \emph{supervisor} (or \emph{advisor}) that provides help for the student and assessment of the student's thesis when it is finished. It gets more complicated, though, when the student chooses a topic of an external party because the person who created it usually is not associated with the university. In this case there needs to be a supervisor that supervises the topic from the perspective of the university and an \emph{external supervisor} that supervises it from the perspective of the external party. Thesis topic can be managed by a university supervisor or by an external supervisor.

These are the main records of a thesis topic that we need to register:
\begin{description}
  \item[Supervisor] -- Either an external supervisor or a university supervisor, provides help and clarifies inaccuracies for the student as far as the topic is concerned
  \item[Title] -- The title of the thesis topic
  \item[Description] -- Thorough description of the topic, it is not necessarily the same as the thesis description
\end{description}

\subsection{Thesis}

Thesis is a project that the student works on at the end of their study and is then assessed by a supervisor and a reader.
The thesis is always managed by a university supervisor because the university supervisor helps the student to follow the university policy.

These are the main records of a thesis that we need to register:
\begin{description}
  \item[Supervisor] -- A university supervisor, assesses the thesis when it is finished
  \item[Reader] -- A reader, assesses the thesis
  \item[Title] -- The title of the thesis
  \item[Description] -- Official description of the thesis
  \item[Abstract] -- A brief summary of the thesis submitted by the student
  \item[Thesis text] -- The thesis text submitted by the student
\end{description}

\subsection{Thesis Management System}

The core project of this thesis is to design and implement a web application for thesis management. In the following chapters, the author will refer to it as the \emph{Thesis Management System} as that is the name of the application.