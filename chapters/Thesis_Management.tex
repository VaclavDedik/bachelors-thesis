\chapter{Thesis Management}

\section{Introduction}

This chapter aims to give a brief description of the thesis management and the related issues. However, the content of this chapter is not to be regarded as a general description of the thesis management, but rather an introduction to the background of the core project of this thesis.

When students fulfill all mandatory subjects and other requirements established by their university, they have to write a \emph{thesis}. To write a thesis, they first have to choose the \emph{topic} of their thesis. There are many ways to choose a topic for a thesis, students can, for example, think of one themselves and contact a teacher at their university. Another way is to choose the topic from the list of topics made by the university or by some external party (e.g. a company that cooperates with the university). A topic usually consists of a \emph{title}, \emph{description} and a \emph{supervisor} who helps the student with the topic (e.g. clarifies inaccuracies).

Thesis, on the other hand, is the piece of work that is based on a topic and consists of an official title and description, supervisor, abstract etc. It is important to note here, however, that the supervisor must be an academic because they need to understand the policies of the university in question. This complicates the thesis management because if an external party is to offer a list of topics for students, they need one of their supervisors, who can help the student with the topic, and a university supervisor, who can help the student to follow the university policies.

The described problem presents us with the fact that we need to differentiate between the topic and the thesis, but most importantly it means that we need to manage them separately.

\section{Terminology}

There is a certain lack of terminology in the field of the thesis management. There is, for example, no one-word term for the external supervisor, who supervises a topic from the point of view of an external party, or the university supervisor, who supervises the topic at a university. For that reason, the authors of the core project of this thesis established their own terminology which is described in this chapter.

As there are two kinds of supervisors, the author calls the external supervisor (i.e. the supervisor from an external party, e.g. a company or an organization) the \emph{leader}. The university supervisor is simply called the \emph{supervisor}. We also need to differentiate between the various types of theses. There is, for example, a diploma thesis (which is, in the context of this thesis, the same as master's thesis) and a bachelor's thesis. This is simply called a \emph{type}. The student who is assigned to a thesis, i.e. who works on it, is called the \emph{assignee}. All other terms related to the thesis management, e.g. title, description, grade etc, remain the same.

\subsection{Thesis Management System}

The core project of this thesis is to design and implement a web application for thesis management. In the following chapters, the author will refer to it as the \emph{Thesis Management System} as that is the name of the application.