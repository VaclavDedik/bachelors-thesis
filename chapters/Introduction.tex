\chapter{Introduction}

In our contemporary society, there is a whole variety of information we need to take care of. Starting with information about people, cars, or real estate and ending with information about furniture or logistics, it is hard to imagine the society developing any further without a means of a fast and easy management of such data. Fortunately, in the age of computer science, it is possible to register and aggregate nearly unlimited amount of information thanks to so-called information systems. The goal of the core project of this thesis is to implement such an information system for the management of theses.

The need for this system was initiated by the Czech subsidiary of the first one-billion dollar open source company\cite{redhat-revenue}, Red Hat. The company currently uses an internal wiki -- a website that allows its users to add, modify, or delete its content. The disadvantages of such a solution include missing support for integrity constraints (anyone can put any information in the system) and authorization. The current solution also requires its users to do substantial amount of management manually in a rather complicated manner. A specialized information system will not only solve these disadvantages, but also open a window for other features, which are exclusive to this particular domain.

In this thesis, the author will describe the way information is handled in terms of theses, and establish the terminology for the core project. He will also cover the most common methods of software development of information systems including the one used for development of the core project. In the second part of the thesis, the author will focus on the architecture and design of the implemented system. The architecture and design chapter will include the list of the functional and non-functional requirements of the system, and mainly the description of the domain model that was designed. This thesis will also cover most of the technologies that were used for the implementation of the project and the last chapter will discuss some possible improvements that could extend the functionality of the system in the future.
